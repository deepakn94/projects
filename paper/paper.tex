\documentclass[11pt,letterpaper,twoside,english]{article}

\usepackage[margin=1.4in]{geometry} % controls the size of the margins

% Special symbols, etc.
\usepackage{amssymb,amsbsy,latexsym}
\usepackage{amsmath}
\usepackage{graphics, subfigure, float} 
\usepackage{cancel}
%\usepackage{todonotes}
% Encoding settings
\usepackage[latin1]{inputenc}
\usepackage[american]{babel}
\usepackage[T1]{fontenc} 

\usepackage{titling} % allows posttitle command

% AMS Math packages

\usepackage{amscd,amsthm}

\usepackage{verbatim, comment} % can comment out text 
\usepackage{mdwlist} 

% Graphics
\usepackage[dvips]{graphicx,epsfig,color}
\usepackage{subfigure}
%\usepackage{pst-all}
%\usepackage{pstricks-add}
\usepackage{hyperref}  % can only be used with pdflatex - gives hyperlinks
\usepackage{bm} % bold math font
\usepackage{bbm}
\usepackage{todonotes}

\newtheoremstyle{theorem}{1em}{1em}{\slshape}{0pt}{\bfseries}{.}{ }{}
\theoremstyle{theorem}
\newtheorem{theorem}{Theorem}
\newtheorem*{theorem*}{Theorem}
\newtheorem{corollary}[theorem]{Corollary}
\newtheorem{proposition}[theorem]{Proposition}
\newtheorem{lemma}[theorem]{Lemma}
\newtheorem{claim}[theorem]{Claim}
\newtheorem{conjecture}[theorem]{Conjecture}
\newtheorem{definition}[theorem]{Definition}
\newtheorem*{claim*}{Claim}

\theoremstyle{remark}
\newtheorem{remark}{Remark}
\newtheorem*{remark*}{Remark}
\newtheorem{algorithm}{Algorithm}
\newtheorem*{question*}{Question}
\newtheorem{question}{Question}
\newtheorem{example}{Example}

\providecommand{\setN}{\mathbb{N}}
\providecommand{\setZ}{\mathbb{Z}}
\providecommand{\setQ}{\mathbb{Q}}
\providecommand{\setR}{\mathbb{R}}
\providecommand{\E}{\mathrm{E}}
\providecommand{\Pr}{\mathrm{Pr}}
\providecommand{\Var}{\mathrm{Var}}

\makeatother

\title{18.821 Project 2} 

\author{Yajit Jain, Deepak Narayanan, Leon Zhang}

\begin{document}

\maketitle

\section{Terms in the sequence $s_n(312)$}

We claim that the sequence of numbers $s_n(312)$ is in fact the sequence of Catalan numbers. We state this result formally as the following theorem,

\begin{theorem}
The total number of permutations of $\{1,2,3,...,n\}$ that avoid the order $312$ as a subsequence is $C_n$ where $C_n$ is the $n^{th}$ Catalan number.
\end{theorem}

Before proving the theorem, we state and prove the following lemma, that will be used in our proof of the theorem.
\begin{lemma}
All permutations of $\{1,2,...,k,k+1\}$ ending in $i$ that avoid the order $312$ as a sub-sequence must be of the form,
$$\pi_1 \pi_2 i$$
where $\pi_1$ is a permutation of $A$ that avoids the order $312$ as a sub-sequence and $\pi_2$ is a permutation of $B$ that avoids the order $312$ as a sub-sequence.
\end{lemma}

\begin{proof}
It is clear that the permutations $\pi_1$ and $\pi_2$ themselves must avoid the order $312$ as a subsequence.

Now, we need to prove that the ordering between $\pi_1$ and $\pi_2$ is necessary. We proceed with a proof by contradiction.

For the sake of contradiction, let us assume that there exists some permutation $\pi$ of $\{1,2,...,k,k+1\}$ that ends with value $i$ such that some integer $x < i$ is to the right of some integer $y > i$. Clearly, this permutation is not of the form described above. It is also easy to see that $\pi$ does \textbf{not} avoid the order $312$ since the triple $(y, x, i)$ satisfies the condition $y > x > i$ and hence is in the order $312$.

\end{proof}

With this lemma proven, we move on to the proof of our theorem.

\begin{proof}
The inductive hypothesis holds for our base case of $\{1\}$, since the only permutation of $\{1\}$ avoids $312$.

Now, we need to prove the inductive case. Let us first assume that for all $i$ from $1$ to $k$, the number of permutations of $\{1,2,...,i\}$ that avoid the order $312$ as a subsequence is $C_i$.

Now, we want to prove the inductive hypothesis for $\{1,2,...,k,k+1\}$ as well, that is the number of permutations of $\{1,2,...,k,k+1\}$ that avoid the order $312$ as a subsequence is $C_{k+1}$.

We count the number of permutations of $\{1,2,...,k,k+1\}$ that avoid $312$ by enumerating through all possible values of the last term of a valid permutation.
If the last term of the permutation is $i$ (where $i \in \{1,2,...,k,k+1\}$), then let us define the subsets $A$ and $B$ of the  set $\{1,2,...,k+1\} \setminus \{i\}$ as the set of integers less than $i$ and the set of integers greater than $i$ respectively. It is clear from the definition of $A$ and $B$ that $A$ and $B$ are disjoint from each other.

Now, from the above lemma, we know that all permutations of $\{1,2,...,k,k+1\}$ ending in $i$ that avoid $312$ must be of the form,

$$\pi = \pi_1 \pi_2 i$$

where $\pi_1$ is a permutation of $A$ that avoids the order $312$ as a sub-sequence and $\pi_2$ is a permutation of $B$ that avoids the order $312$ as a sub-sequence. It is clear that the above permutation contains all integers between $1$ and $k+1$, from the definitions of the subsets $A$ and $B$, which implies that $\pi_1\pi_2i$ is permutation of the set $\{1,2,...,k,k+1\}$.

Now, the total number of permutations $\pi$ is,
$$n_{\pi_1} \cdot n_{\pi_2} = C_{i-1} \cdot C_{k-i+1}$$
since the total number of valid permuations $\pi_1$ is simply going to be $C_{i-1}$ (total number of valid permutations of length $i-1$ that avoid the order $312$ as a sub-sequence is $C_{i-1}$, and correspondingly $n_{\pi_2} = C_{k-1+1}$)

Now, summing over all valid $i$, we see that the total number of permutations of $\{1,2,...,k+1\}$ that avoid $312$ is equal to,
$$\sum_{i=1}^{k+1} C_{i-1} \cdot C_{k-i+1} = \sum_{i=0}^k C_i \cdot C_{k-i}$$ which is in fact $C_{k+1}$, and we are done.

\end{proof}

\end{document}