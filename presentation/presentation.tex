\documentclass{beamer}
\usepackage{tkz-graph}
\usepackage{fancyvrb}
\usepackage{amsmath}
\usetheme{Amsterdam}

\usepackage{verbatim}
\usetikzlibrary{arrows,shapes}

\makeatletter
\newenvironment<>{proofwithoutbox}[1][\proofname]{%
    \par
    \def\insertproofname{#1\@addpunct{.}}%
    \usebeamertemplate{proof begin}#2}
  {\usebeamertemplate{proof end}}
\makeatother

\title[Pattern avoidance]{Pattern avoidance}
\subtitle{An explanation and proof}
\author[Jain, Narayanan and Zhang]{Yajit Jain, Deepak Narayanan and Leon Zhang}
\date[November 2014]{November 14, 2014}

\RecustomVerbatimCommand{\VerbatimInput}{VerbatimInput}%
{fontsize=\tiny,
 commentchar=*        % comment character
}

\begin{document}

\pgfdeclarelayer{background}
\pgfsetlayers{background,main}

\tikzstyle{vertex}=[circle,fill=black!25,minimum size=20pt,inner sep=0pt]
\tikzstyle{selected vertex} = [vertex, fill=red!24]
\tikzstyle{completed vertex} = [vertex, fill=yellow]
\tikzstyle{edge} = [draw,thick,-]
\tikzstyle{weight} = [font=\small]
\tikzstyle{selected edge} = [draw,line width=5pt,-,red!50]
\tikzstyle{ignored edge} = [draw,line width=5pt,-,black!20]

\begin{frame}[plain]
  \titlepage
\end{frame}

\begin{frame}{Plan of action}
    \begin{enumerate}
        \item Introduction
        \item Avoidance in $S_n$
        \item Avoidance in $T_n$
    \end{enumerate}
\end{frame}


\begin{frame}{}
\begin{definition}
A finite sequence of distinct positive integers $a_1a_2\cdots a_n$ avoids another finite sequence of distinct positive integers $b_1\cdots b_k$ with $n\ge k$ if no subsequence $a_{i_1}a_{i_2}\cdots a_{i_k}$ of $a_1a_2\cdots a_n$ has its terms in the same relative order as $b_1\cdots b_k$. 
\end{definition}

\

\

\

Examples!
\end{frame}

\begin{frame}
Given $k$ and $n$, how many permutations of the set $\{1,\cdots, n\}$ avoid a given permutation of the set $\{1,\cdots, k\}$?\

\

\begin{definition}
Let $\pi$ be a permutation of length $k$. Then
$$
s_n(\pi)=\#\{\text{the set of permutations of $\{1,\cdots, n\}$ that avoid $\pi$}\}.
$$
\end{definition}

\

{\color{blue} We are going to compute sequences $(s_n(\pi))$}.
\end{frame}

\begin{frame}{Formalities}
\begin{definition}
A permutation of a set $D$ is a bijective map from $D$ to itself. 
\end{definition}
\

\

\

\begin{definition}
$S_n$ is the {\color{red}set} of permutations of $\{1,\cdots, n\}$. 
\end{definition}
\end{frame}

\begin{frame}{Example}
Take $\sigma\in S_5$ with
$$
\sigma=\begin{pmatrix}
1&2&3&4&5\\
\downarrow&\downarrow&\downarrow&\downarrow&\downarrow&\\
4&5&2&1&3
\end{pmatrix}
$$

\

\

\

\begin{columns}
\begin{column}{.5\textwidth}
\centering{\underline{One-line Notation}}
$$
\sigma=45213
$$
\end{column}
\vrule{}
\begin{column}{.5\textwidth}
\centering{\underline{Cycle Notation}}
$$
\sigma=(14)(253)
$$
\end{column}
\end{columns}

\end{frame}

%%%%

\begin{frame}{Example}
Take $\sigma\in S_5$ with
$$
\sigma=\begin{pmatrix}
1&2&3&4&5\\
\downarrow&\downarrow&\downarrow&\downarrow&\downarrow&\\
\color{red}4&\color{red}5&\color{red}2&\color{red}1&\color{red}3
\end{pmatrix}
$$

\

\

\

\begin{columns}
\begin{column}{.5\textwidth}
\centering{\underline{One-line Notation}}
\color{red}$$
\sigma=45213
$$
\end{column}
\vrule{}
\begin{column}{.5\textwidth}
\centering{\underline{Cycle Notation}}
$$
\sigma=(14)(253)
$$
\end{column}
\end{columns}

\end{frame}

%%%%%

\begin{frame}{Example}
Take $\sigma\in S_5$ with
$$
\sigma=\begin{pmatrix}
1&2&3&4&5\\
\color{red}\downarrow&\downarrow&\downarrow&\color{red}\downarrow&\downarrow&\\
4&5&2&1&3
\end{pmatrix}
$$

\

\

\

\begin{columns}
\begin{column}{.5\textwidth}
\centering{\underline{One-line Notation}}
$$
\sigma=45213
$$
\end{column}
\vrule{}
\begin{column}{.5\textwidth}
\centering{\underline{Cycle Notation}}
$$
\sigma={\color{red}(14)}(253)
$$
\end{column}
\end{columns}

\end{frame}


%%%%%%

\begin{frame}{Example}
Take $\sigma\in S_5$ with
$$
\sigma=\begin{pmatrix}
1&2&3&4&5\\
\color{red}\downarrow&\color{green}\downarrow&\color{green}\downarrow&\color{red}\downarrow&\color{green}\downarrow&\\
4&5&2&1&3
\end{pmatrix}
$$

\

\

\

\begin{columns}
\begin{column}{.5\textwidth}
\centering{\underline{One-line Notation}}
$$
\sigma=45213
$$
\end{column}
\vrule{}
\begin{column}{.5\textwidth}
\centering{\underline{Cycle Notation}}
$$
\sigma={\color{red}(14)}{\color{green}(253)}
$$
\end{column}
\end{columns}

\end{frame}

%%%%%%

\begin{frame}{\color{red}Main Result}
\begin{definition}[Recall]
Let $\pi$ be a permutation of length $k$. Then
$$
s_n(\pi)=\#\{\text{the set of permutations in $S_n$ that avoid $\pi$}\}.
$$
\end{definition}

\


\begin{Theorem}
For $\pi\in S_3$, $(s_n(\pi))$ is equal to the Catalan numbers:
$$
(s_n(\pi))=1,1,2,5,14,42,132,429...
$$
\end{Theorem}
\end{frame}

%%%%%%


\begin{frame}{Conjectures for $\pi\in S_4$}
$$
(s_n(\pi))=\left\{
	\begin{array}{ll}
A:=1,2,6,23,103,512,2740,15485,91245\ldots\\
B:=1,2,6,23,103,513,2761,15767,94359\ldots\\
C:=1,2,6,23,103,513,2762,15793,94776\ldots
	\end{array}
\right\}
$$

\begin{center}
\begin{tabular}{|c|c|c|}
B &A&C\\
\hline
(1)(2)(3)(4),(14)(23)&(243),(142)&(23),(14)\\
(34),(1423)&(234),(143)&\\
(24),(1432)&(124),(132)&\\
(12),(1324)&(123),(134)&\\
(12)(34),(13)(24)&(1243),(1342)&\\
(1234),(13)&&\\
\end{tabular}
\end{center}

\end{frame}

%%%%%%
\begin{frame}
\begin{definition}[Reversing]
We define the \emph{reverse} of a sequence $a_1\cdots a_n$ to be the sequence $a_n\cdots a_1$.  The flipping operator is denoted by $\mathcal{R}$.  
\end{definition}

\begin{example}
$\mathcal{R}(1324)=4231$.
\end{example}

\begin{example}
$\mathcal{R}(1243)=3421$.
\end{example}

\end{frame}

\begin{frame}

\begin{lemma}[Reversing Lemma]
The permutation $\sigma$ avoids the permutation $\pi$ iff $\mathcal{R}(\sigma)$ avoids $\mathcal{R}(\pi)$.
\end{lemma}
\begin{proof}
Need to only prove one direction (why?).

Suppose that $\mathcal{R}(\sigma)$ does not avoid $\mathcal{R}(\pi)$ if $\sigma$ avoids $\pi$.

Reversing $\mathcal{R}(\sigma)$ again produces a contradiction (why?), which implies that $\mathcal{R}(\sigma)$ avoids $\mathcal{R}(\pi)$ if $\sigma$ avoids $\pi$.

\end{proof}

\begin{corollary} 
For a permutation $\pi$, $s_n(\pi)=s_n(\mathcal{R}(\pi))$. 
\end{corollary}

\end{frame}



\begin{frame}
\begin{definition}[Flip of a sequence]
We define the \emph{flip} of a sequence $a$ as the sequence $b$ with the same elements as $a$, but with the largest element swapped with the smallest element, the second largest element swapped with the second smallest element, etc.  The flipping operator is denoted by $\mathcal{F}$.  
\end{definition}

\begin{example}
$\mathcal{F}(1324)=4231$.
\end{example}

\begin{example}
$\mathcal{F}(1243)=4312$.
\end{example}

\end{frame}

\begin{frame}
\begin{definition}[Flip of a sequence]
We define the \emph{flip} of a sequence $a$ as the sequence $b$ with the same elements as $a$, but with the largest element swapped with the smallest element, the second largest element swapped with the second smallest element, etc.  The flipping operator is denoted by $\mathcal{F}$.  
\end{definition}

\begin{example}
$\mathcal{F}(1324)=4231$. {\color{red} $\mathcal{R}(1324)=\mathcal{F}(1324)$}
\end{example}

\begin{example}
$\mathcal{F}(1243)=4312$. {\color{red} $\mathcal{R}(1243)\neq\mathcal{F}(1243)$}
\end{example}

\end{frame}

\begin{frame}
\begin{lemma}[Flipping Lemma]
The permutation $\sigma$ avoids the permutation $\pi$ iff $\mathcal{F}(\sigma)$ avoids $\mathcal{F}(\pi)$.
\end{lemma}

\begin{proof}
As before, need to only prove one direction.

Suppose $\mathcal{F}(\sigma)$ does not avoid $\mathcal{F}(\pi)$ if $\sigma$ avoids $\pi$. (Here, $\sigma \in S_n$ and $\pi \in S_k$)

Let $\mathcal{F}(\sigma)=a_1\cdots a_n$ and let $\mathcal{F}(\pi)=b_1\cdots  b_k$. Then there is some subsequence $a_{i_1}\cdots a_{i_k}$ such that $a_{i_c}<a_{i_d}$ iff $b_c<b_d$.

But then, $n+1-a_{i_c}>n+1-a_{i_d}$ iff $k+1-b_c>k+1-b_d$. This implies that $\sigma$ does not avoid $\pi$, a contradiction (why?). 
\end{proof}

\begin{corollary} 
For a permutation $\pi$, $s_n(\pi)=s_n(\mathcal{F}(\pi))$. 
\end{corollary}

\end{frame}

\begin{frame}
\begin{lemma}
The permutations of $\{1,2,\dots,k,k+1\}$ ending in $i$ that avoid the pattern $312$ are precisely those of the form,
$$\pi_1 \pi_2 i$$
the concatenation of $\pi_1, \pi_2$, and $i$, where $\pi_1$ is a permutation of $\{1,2,\ldots,i-1\}$ that avoids the pattern $312$ and $\pi_2$ is a permutation of $\{i+1,\ldots,k+1\}$ that avoids the pattern $312$.
\end{lemma}
\end{frame}

\begin{frame}
\begin{proof}

Assume that $\exists$ some permutation $\pi$ of $\{1,2,...,k,k+1\}$ that avoids $312$, ending with value $i$ such that some integer $x < i$ is to the right of some integer $y > i$. Clearly, $\pi$ does not avoid 312. (why?)

It follows that for any permutation $\pi$ avoiding $312$ and ending in $i$, all $x < i$ must be to the left of all $y > i$. Hence, $\pi$ can be written as $$\pi_1 \pi_2 i$$ where $\pi_1$ is a permutation of $\{1,2,\ldots,i-1\}$ and $\pi_2$ is a permutation of $\{i+1, i+2,\ldots,k+1\}$.

Furthermore, it is clear that the permutations $\pi_1$ and $\pi_2$ must avoid $312$ as well. (why?)
\end{proof}

\end{frame}

\begin{frame}
\begin{definition}
The Catalan numbers are the sequence of positive integers $C_i$ defined as follows,
$$C_0=1, \, C_{n+1}=\sum_{i=0}^n C_iC_{n-i} \, \text{for} \; n \geq 0$$
\end{definition}

\begin{theorem}
$s_n(312), s_n(132), s_n(213)$, and $s_n(231)$ are equal to $C_n$, the $n^{th}$ Catalan number.
\end{theorem}
\end{frame}

\begin{frame}
\begin{proofwithoutbox}
Assume that for all $i$ from $1$ to $k$, the number of permutations of $\{1,2,...,i\}$ that avoid the order $312$ as a subsequence is $C_i$.

The number of permutations of $\{1,2,...,k,k+1\}$ that avoid $312$ can be counted by enumerating through all possible values of the last term of a valid permutation. 

From the above lemma, all permutations of $\{1,2,...,k,k+1\}$ ending in $i$ that avoid $312$ are precisely those of the form
$$\pi = \pi_1 \pi_2 i$$
where $\pi_1$ is a permutation of $\{1,2,\dots,i-1\}$ that avoids $312$ and $\pi_2$ is a permutation of $\{i+1,i+2,\ldots,k+1\}$ that avoids $312$.
\end{proofwithoutbox}
\end{frame}

\begin{frame}
\begin{proof}[\proofname\ (contd.)]
It follows that the total number of permutations $\pi$ avoiding $312$ and ending in $i$ is
$$C_{i-1} \cdot C_{k-i+1}$$

Summing over all possible values of $i$, the total number of permutations of $\{1,2,...,k+1\}$ that avoid $312$ is equal to,
$$\sum_{i=1}^{k+1} C_{i-1} \cdot C_{k-i+1} = \sum_{i=0}^k C_i \cdot C_{k-i} = C_{k+1}$$

By the flipping lemma and reversing lemma,  $s_n(132)$, $s_n(213)$ and $s_n(231)$ are the sequence of Catalan numbers as well. 
\end{proof}
\end{frame}


\begin{frame}{Avoidance in $T_n$}
\begin{definition}
Let $n=2m$. Then
\begin{align*}T_n= &\{\sigma \in S_n\, |\ 1,3,5,\ldots, 2n-1\text{ appear in increasing}\\ \ & \text{ order, and }2i\text{ is always to the right of 2i-1}\}.
\end{align*}
\end{definition}
\begin{example}
The set $T_2\subset S_2$ consists of the single permutation $12$.
\end{example}
\end{frame}

\begin{frame}
\begin{example}
The set $T_4\subset S_4$ consists of the permutations $1234, 1324,$ and $1342$.
\end{example}
\begin{example}
The set $T_6\subset S_6$ consists of the following permutations:
\begin{align*}
T_6=&\{
123564, 123456, 123546,
132564, 132456, 132546,\\
\ & 135264,  134256, 135246,
135624, 134526, 135426,\\
\ & 135642, 134562, 135462\}.
\end{align*}
\end{example}
\end{frame}

\begin{frame}
\begin{theorem}
Recall that $n=2m$. The set $T_n$ has size
\[\# T_n=1\cdot 3 \cdot 5 \cdot \ldots \cdot 2m-1.\]
\end{theorem}
\begin{proof}
\begin{enumerate}
\item $\# T_n$ given by the number of valid ways to insert $2, 4, \ldots, 2m$ into the sequence $1, 3, \ldots, 2m-1$.
\item Insert in reverse order: first $2m$, then $2m-2$, and so on.
\item Only one way to insert $2m$. Then three ways to insert $2m-2$. Then five ways to insert $2m-4$. Repeat... there are $2m-1$ ways to insert $2$.
\end{enumerate}
\end{proof}
\end{frame}

\begin{frame}
\begin{lemma}
Let $\sigma\in T_n$, and let $a, b\in \{1, 2, \ldots, n\}$ with $a$ to the left of $b$ in $\sigma$. If $a>b$, then $b$ is even.
\end{lemma}
\begin{proof}
Two possibilities for $a$:
\begin{itemize}
\item $a$ is odd. Then $b$ is not odd, as odd integers must appear in increasing order.
\item $a$ is even. Then 
\begin{itemize}
\item$a-1$ is to the left of $a$, so to the left of $b$
\item Since $a-1\geq b$, and $a-1\neq b$, we have $a-1> b.$
\end{itemize}
and we have reduced to the previous case.
\end{itemize}
\end{proof}
\end{frame}


\begin{frame}{Problem Statement}
\begin{definition}
Given a permutation $\pi\in S_k$, we define $t_n(\pi)$ as
\[t_n(\pi)=\#\{\sigma\in T_n\, | \ \sigma\text{ avoids } \pi\}.\]
\end{definition}
\begin{problem}
Let $\pi\in S_3$, and $n$ an arbitrary positive integer. Compute $t_n(\pi)$.
\end{problem}
\end{frame}

\begin{frame}
We can run code to compute $t_n(\pi)$ for small $n$ and for each $\pi\in S_3$. We get
\[\begin{tabular}{|c | c | c | c | c | c|}
\hline$\pi$ & $n=2$ & $n=4$ & $n=6$ & $n=8$ & $n=10$\\
\hline
123 & 1 & 0 & 0 & 0 & 0\\
\hline
132 & 1 & 1 & 1 & 1 & 1\\
\hline
213 & 1 & 2 & 4 & 8 & 16\\
\hline
231 & 1 & 2 & 4 & 8 & 16\\
\hline
312 & 1 & 3 & 12 & 55 & 273\\
\hline
321 & 1 & 3 & 12 & 55 & 273\\\hline
\end{tabular}\]
\end{frame}

\begin{frame}
\begin{itemize}
\item Easy to see that $t_n(123)=0$ when $n\geq 2$
\begin{itemize}
\item The subsequence 134 is always present in a permutation $\sigma\in t_n(123)$.
\end{itemize}
\item Also easy to see that $t_n(132)=1$. 
\begin{itemize}
\item The permutation $123\ldots (2n)$ is the only permutation in $T_n$ that avoids 132.
\end{itemize}
\item What about the other permutations in $S_3$?
\end{itemize}
\end{frame}

\begin{frame}
\begin{theorem}
The total number of permutations in $T_n$ avoiding 231, $t_n(231)$, is equal to $2^{\frac{n}{2}-1}$.
\end{theorem}
\begin{proofwithoutbox}[Proof outline]
Fix an arbitrary $\sigma\in T_n$. Consider an arbitrary $2i\in\{1, 2, \ldots, n\}$, and define
\[A=\{a\in \{1, 2, \ldots, n\}\, | \ a > 2i \text{ and } a \text{ to the left of 2i}\}.\]
Proof steps:
\begin{itemize}
\item The set $A$ contains no even numbers.
\item The set $A$ contains no odd numbers greater than $2i+1$.
\item Any integer to the left of $2i$ must be less than $2i$, unless it is $2i+1$.
\end{itemize}
\end{proofwithoutbox}
\end{frame}

\begin{frame}
\begin{proof}[Proof outline (Continued).]
\begin{itemize}
\item Use induction to show that either 
\begin{itemize}
\item $2i$ is in the $2i$th position, with $2i+1$ in the $2i+1$st position, or
\item $2i+1$ is in the $2i$th position, with $2i$ in the $2i+1$st position.
\end{itemize}
\item Conclude that there are $2^{\frac{n}{2}-1}$ options for $\sigma$.
\end{itemize}
\end{proof}
\end{frame}

\begin{frame}{Other results and conjectures}
\begin{theorem}
The total number of permutations in $T_n$ avoiding 213, $t_n(213)$, is equal to $2^{\frac n 2-1}$.
\end{theorem}
\begin{theorem}
There exists a bijection between the set of permutations in $T_n$ avoiding 312 and the set of permutations in $T_n$ avoiding 321. As a consequence, $t_n(312)=t_n(321)$.
\end{theorem}
\begin{proofwithoutbox}[Conjecture]
The total number of permutations in $T_n$ avoiding 312 or 321, $t_n(312)$ and $t_n(321)$ respectively, is given by the equation $\binom{3n/2}{n/2}/(n+1)$.
\end{proofwithoutbox}
\end{frame}

\end{document}