\documentclass{beamer}
\usepackage{tkz-graph}
\usepackage{fancyvrb}
\usetheme{Amsterdam}

\usepackage{verbatim}
\usetikzlibrary{arrows,shapes}

\makeatletter
\newenvironment<>{proofwithoutbox}[1][\proofname]{%
    \par
    \def\insertproofname{#1\@addpunct{.}}%
    \usebeamertemplate{proof begin}#2}
  {\usebeamertemplate{proof end}}
\makeatother

\title[Avoidance in permutations]{Pattern avoidance}
\subtitle{An explanation and proof}
\author[Jain, Narayanan and Zhang]{Yajit Jain, Deepak Narayanan and Leon Zhang}
\date[November 2014]{November 14, 2014}

\RecustomVerbatimCommand{\VerbatimInput}{VerbatimInput}%
{fontsize=\tiny,
 commentchar=*        % comment character
}

\begin{document}

\pgfdeclarelayer{background}
\pgfsetlayers{background,main}

\tikzstyle{vertex}=[circle,fill=black!25,minimum size=20pt,inner sep=0pt]
\tikzstyle{selected vertex} = [vertex, fill=red!24]
\tikzstyle{completed vertex} = [vertex, fill=yellow]
\tikzstyle{edge} = [draw,thick,-]
\tikzstyle{weight} = [font=\small]
\tikzstyle{selected edge} = [draw,line width=5pt,-,red!50]
\tikzstyle{ignored edge} = [draw,line width=5pt,-,black!20]

\begin{frame}[plain]
  \titlepage
\end{frame}

\begin{frame}{Plan of action}
    \begin{enumerate}
        \item Definition, Introduction and Motivation
        \item Avoidance in $S_n$
        \item Avoidance in $T_n$
    \end{enumerate}
\end{frame}


\begin{frame}{Definition, Introduction and Motivation}

\end{frame}


\begin{frame}

\begin{lemma}[Reversing Lemma]
The permutation $\sigma$ avoids the permutation $\pi$ if and only if $\mathcal{R}(\sigma)$ avoids $\mathcal{R}(\pi)$.
\end{lemma}
\begin{proof}
Need to only prove one direction (why?).

Suppose that $\mathcal{R}(\sigma)$ does not avoid $\mathcal{R}(\pi)$ if $\sigma$ avoids $\pi$.

Then if $\mathcal{R}(\sigma)$ is reversed again again, the subsequence of $\mathcal{R}(\sigma)$ that follows the same pattern as $\mathcal{R}(\pi)$ will reverse to a subsequence that follows the same pattern as $\pi$ -- a contradiction. Therefore $\mathcal{R}(\sigma)$ avoids $\mathcal{R}(\pi)$. 

\end{proof}

\begin{corollary} 
For a permutation $\pi$, $s_n(\pi)=s_n(\mathcal{R}(\pi))$. 
\end{corollary}

\end{frame}


\begin{frame}
\begin{definition}
We define the \emph{flip} of a sequence $a$ as the sequence $b$ with the same elements as $a$, but with the largest element swapped with the smallest element, the second largest element swapped with the second smallest element, etc.  The flipping operator will be denoted by $\mathcal{F}$.  
\end{definition}

\begin{example}
$\mathcal{F}(1324)=4231$.
\end{example}

\begin{example}
$\mathcal{F}(1243)=4312$.
\end{example}

\end{frame}

\begin{frame}
\begin{lemma}[Flipping Lemma]
The permutation $\sigma$ avoids the permutation $\pi$ if and only if $\mathcal{F}(\sigma)$ avoids $\mathcal{F}(\pi)$.
\end{lemma}

\begin{proof}
Need to only prove one direction (why?).

Suppose $\mathcal{F}(\sigma)$ does not avoid $\mathcal{F}(\pi)$ if $\sigma$ avoids $\pi$. (Here, $\sigma \in S_n$ and $\pi \in S_k$)

Let $\mathcal{F}(\sigma)=a_1\cdots a_n$ and let $\mathcal{F}(\pi)=b_1\cdots  b_k$. Then there is some subsequence $a_{i_1}\cdots a_{i_k}$ so that $a_{i_c}<a_{i_d}$ if and only if $b_c<b_d$.

But then, $n+1-a_{i_c}>n+1-a_{i_d}$ if and only if $k+1-b_c>k+1-b_d$. This implies that $\sigma$ does not avoid $\pi$, a contradiction (why?). 
\end{proof}

\begin{corollary} 
For a permutation $\pi$, $s_n(\pi)=s_n(\mathcal{F}(\pi))$. 
\end{corollary}

\end{frame}

\begin{frame}
\begin{lemma}
The permutations of $\{1,2,\dots,k,k+1\}$ ending in $i$ that avoid the pattern $312$ are precisely those of the form,
$$\pi_1 \pi_2 i$$
the concatenation of $\pi_1, \pi_2$, and $i$, where $\pi_1$ is a permutation of $\{1,2,\ldots,i-1\}$ that avoids the pattern $312$ and $\pi_2$ is a permutation of $\{i+1,\ldots,k+1\}$ that avoids the pattern $312$.
\end{lemma}
\end{frame}

\begin{frame}
\begin{proof}

For the sake of contradiction, assume that there exists some permutation $\pi$ of $\{1,2,...,k,k+1\}$ that avoids $312$, ending with value $i$ such that some integer $x < i$ is to the right of some integer $y > i$. Clearly, $\pi$ does not avoid 312. (why?)

It follows that for any permutation $\pi\in S_{k+1}$ avoiding $312$ and ending in $i$, all $x < i$ must be to the left of all $y > i$. Hence, $\pi$ can be written as $\pi_1 \pi_2 i$ where $\pi_1$ is a permutation of $\{1,2,\ldots,i-1\}$ and $\pi_2$ is a permutation of $\{i+1, i+2,\ldots,k+1\}$. Furthermore, it is clear that any subsequences of the permutation $\pi$ must avoid $312$ if the entire permutation $\pi$ is to avoid $312$ as well; this implies that the permutations $\pi_1$ and $\pi_2$ must avoid $312$ as well.
\end{proof}

\end{frame}

\begin{frame}
\begin{definition}
The Catalan numbers are the sequence of positive integers $C_i$ defined as follows,
$$C_0=1, \, C_{n+1}=\sum_{i=0}^n C_iC_{n-i} \, \text{for} \; n \geq 0$$
\end{definition}

\begin{theorem}
$s_n(312), s_n(132), s_n(213)$, and $s_n(231)$ are equal to $C_n$, the $n^{th}$ Catalan number.
\end{theorem}
\end{frame}

\begin{frame}
\begin{proofwithoutbox}
Assume that for all $i$ from $1$ to $k$, the number of permutations of $\{1,2,...,i\}$ that avoid the order $312$ as a subsequence is $C_i$.

Want to prove that the number of permutations of $\{1,2,...,k,k+1\}$ that avoid the order $312$ as a subsequence is $C_{k+1}$.

The number of permutations of $\{1,2,...,k,k+1\}$ that avoid $312$ can be counted by enumerating through all possible values of the last term of a valid permutation. 

From the above lemma, all permutations of $\{1,2,...,k,k+1\}$ ending in $i$ that avoid $312$ are precisely those of the form
$$\pi = \pi_1 \pi_2 i$$
where $\pi_1$ is a permutation of $\{1,2,\dots,i-1\}$ that avoids $312$ and $\pi_2$ is a permutation of $\{i+1,i+2,\ldots,k+1\}$ that avoids $312$.
\end{proofwithoutbox}
\end{frame}

\begin{frame}
\begin{proof}[\proofname\ (contd.)]
It follows that the total number of permutations $\pi$ avoiding $312$ and ending in $i$ is
$$C_{i-1} \cdot C_{k-i+1}$$ (by our induction hypothesis)

Now, summing over all possible values of $i$, the total number of permutations of $\{1,2,...,k+1\}$ that avoid $312$ is equal to,
$$\sum_{i=1}^{k+1} C_{i-1} \cdot C_{k-i+1} = \sum_{i=0}^k C_i \cdot C_{k-i}$$ which equals $C_{k+1}$

By the flipping lemma and reversing lemma,  $s_n(132)$, $s_n(213)$ and $s_n(231)$ are the sequence of Catalan numbers as well. 
\end{proof}
\end{frame}


\begin{frame}{Avoidance in $T_n$}

\end{frame}

\end{document}