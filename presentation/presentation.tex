\documentclass{beamer}
\usepackage{tkz-graph}
\usepackage{fancyvrb}
\usetheme{Amsterdam}

\usepackage{verbatim}
\usetikzlibrary{arrows,shapes}

\title[Avoidance in permutations]{Avoidance in permutations}
\subtitle{An explanation and proof}
\author[Jain, Narayanan and Zhang]{Yajit Jain, Deepak Narayanan and Leon Zhang}
\date[November 2014]{November 14, 2014}

\RecustomVerbatimCommand{\VerbatimInput}{VerbatimInput}%
{fontsize=\tiny,
 commentchar=*        % comment character
}

\begin{document}

\pgfdeclarelayer{background}
\pgfsetlayers{background,main}

\tikzstyle{vertex}=[circle,fill=black!25,minimum size=20pt,inner sep=0pt]
\tikzstyle{selected vertex} = [vertex, fill=red!24]
\tikzstyle{completed vertex} = [vertex, fill=yellow]
\tikzstyle{edge} = [draw,thick,-]
\tikzstyle{weight} = [font=\small]
\tikzstyle{selected edge} = [draw,line width=5pt,-,red!50]
\tikzstyle{ignored edge} = [draw,line width=5pt,-,black!20]

\begin{frame}[plain]
  \titlepage
\end{frame}

\begin{frame}{Plan of action}
    \begin{enumerate}
        \item Definition, Introduction and Motivation
        \item Avoidance in $S_n$
        \item Avoidance in $T_n$
    \end{enumerate}
\end{frame}


\begin{frame}{Definition, Introduction and Motivation}

\end{frame}


\begin{frame}{Avoidance in $S_n$}

\end{frame}


\begin{frame}{Avoidance in $T_n$}

\end{frame}

\end{document}